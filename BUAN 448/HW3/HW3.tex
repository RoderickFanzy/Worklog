% Options for packages loaded elsewhere
\PassOptionsToPackage{unicode}{hyperref}
\PassOptionsToPackage{hyphens}{url}
%
\documentclass[
]{article}
\usepackage{amsmath,amssymb}
\usepackage{lmodern}
\usepackage{iftex}
\ifPDFTeX
  \usepackage[T1]{fontenc}
  \usepackage[utf8]{inputenc}
  \usepackage{textcomp} % provide euro and other symbols
\else % if luatex or xetex
  \usepackage{unicode-math}
  \defaultfontfeatures{Scale=MatchLowercase}
  \defaultfontfeatures[\rmfamily]{Ligatures=TeX,Scale=1}
\fi
% Use upquote if available, for straight quotes in verbatim environments
\IfFileExists{upquote.sty}{\usepackage{upquote}}{}
\IfFileExists{microtype.sty}{% use microtype if available
  \usepackage[]{microtype}
  \UseMicrotypeSet[protrusion]{basicmath} % disable protrusion for tt fonts
}{}
\makeatletter
\@ifundefined{KOMAClassName}{% if non-KOMA class
  \IfFileExists{parskip.sty}{%
    \usepackage{parskip}
  }{% else
    \setlength{\parindent}{0pt}
    \setlength{\parskip}{6pt plus 2pt minus 1pt}}
}{% if KOMA class
  \KOMAoptions{parskip=half}}
\makeatother
\usepackage{xcolor}
\usepackage[margin=1in]{geometry}
\usepackage{color}
\usepackage{fancyvrb}
\newcommand{\VerbBar}{|}
\newcommand{\VERB}{\Verb[commandchars=\\\{\}]}
\DefineVerbatimEnvironment{Highlighting}{Verbatim}{commandchars=\\\{\}}
% Add ',fontsize=\small' for more characters per line
\usepackage{framed}
\definecolor{shadecolor}{RGB}{248,248,248}
\newenvironment{Shaded}{\begin{snugshade}}{\end{snugshade}}
\newcommand{\AlertTok}[1]{\textcolor[rgb]{0.94,0.16,0.16}{#1}}
\newcommand{\AnnotationTok}[1]{\textcolor[rgb]{0.56,0.35,0.01}{\textbf{\textit{#1}}}}
\newcommand{\AttributeTok}[1]{\textcolor[rgb]{0.77,0.63,0.00}{#1}}
\newcommand{\BaseNTok}[1]{\textcolor[rgb]{0.00,0.00,0.81}{#1}}
\newcommand{\BuiltInTok}[1]{#1}
\newcommand{\CharTok}[1]{\textcolor[rgb]{0.31,0.60,0.02}{#1}}
\newcommand{\CommentTok}[1]{\textcolor[rgb]{0.56,0.35,0.01}{\textit{#1}}}
\newcommand{\CommentVarTok}[1]{\textcolor[rgb]{0.56,0.35,0.01}{\textbf{\textit{#1}}}}
\newcommand{\ConstantTok}[1]{\textcolor[rgb]{0.00,0.00,0.00}{#1}}
\newcommand{\ControlFlowTok}[1]{\textcolor[rgb]{0.13,0.29,0.53}{\textbf{#1}}}
\newcommand{\DataTypeTok}[1]{\textcolor[rgb]{0.13,0.29,0.53}{#1}}
\newcommand{\DecValTok}[1]{\textcolor[rgb]{0.00,0.00,0.81}{#1}}
\newcommand{\DocumentationTok}[1]{\textcolor[rgb]{0.56,0.35,0.01}{\textbf{\textit{#1}}}}
\newcommand{\ErrorTok}[1]{\textcolor[rgb]{0.64,0.00,0.00}{\textbf{#1}}}
\newcommand{\ExtensionTok}[1]{#1}
\newcommand{\FloatTok}[1]{\textcolor[rgb]{0.00,0.00,0.81}{#1}}
\newcommand{\FunctionTok}[1]{\textcolor[rgb]{0.00,0.00,0.00}{#1}}
\newcommand{\ImportTok}[1]{#1}
\newcommand{\InformationTok}[1]{\textcolor[rgb]{0.56,0.35,0.01}{\textbf{\textit{#1}}}}
\newcommand{\KeywordTok}[1]{\textcolor[rgb]{0.13,0.29,0.53}{\textbf{#1}}}
\newcommand{\NormalTok}[1]{#1}
\newcommand{\OperatorTok}[1]{\textcolor[rgb]{0.81,0.36,0.00}{\textbf{#1}}}
\newcommand{\OtherTok}[1]{\textcolor[rgb]{0.56,0.35,0.01}{#1}}
\newcommand{\PreprocessorTok}[1]{\textcolor[rgb]{0.56,0.35,0.01}{\textit{#1}}}
\newcommand{\RegionMarkerTok}[1]{#1}
\newcommand{\SpecialCharTok}[1]{\textcolor[rgb]{0.00,0.00,0.00}{#1}}
\newcommand{\SpecialStringTok}[1]{\textcolor[rgb]{0.31,0.60,0.02}{#1}}
\newcommand{\StringTok}[1]{\textcolor[rgb]{0.31,0.60,0.02}{#1}}
\newcommand{\VariableTok}[1]{\textcolor[rgb]{0.00,0.00,0.00}{#1}}
\newcommand{\VerbatimStringTok}[1]{\textcolor[rgb]{0.31,0.60,0.02}{#1}}
\newcommand{\WarningTok}[1]{\textcolor[rgb]{0.56,0.35,0.01}{\textbf{\textit{#1}}}}
\usepackage{graphicx}
\makeatletter
\def\maxwidth{\ifdim\Gin@nat@width>\linewidth\linewidth\else\Gin@nat@width\fi}
\def\maxheight{\ifdim\Gin@nat@height>\textheight\textheight\else\Gin@nat@height\fi}
\makeatother
% Scale images if necessary, so that they will not overflow the page
% margins by default, and it is still possible to overwrite the defaults
% using explicit options in \includegraphics[width, height, ...]{}
\setkeys{Gin}{width=\maxwidth,height=\maxheight,keepaspectratio}
% Set default figure placement to htbp
\makeatletter
\def\fps@figure{htbp}
\makeatother
\setlength{\emergencystretch}{3em} % prevent overfull lines
\providecommand{\tightlist}{%
  \setlength{\itemsep}{0pt}\setlength{\parskip}{0pt}}
\setcounter{secnumdepth}{-\maxdimen} % remove section numbering
\ifLuaTeX
  \usepackage{selnolig}  % disable illegal ligatures
\fi
\IfFileExists{bookmark.sty}{\usepackage{bookmark}}{\usepackage{hyperref}}
\IfFileExists{xurl.sty}{\usepackage{xurl}}{} % add URL line breaks if available
\urlstyle{same} % disable monospaced font for URLs
\hypersetup{
  pdftitle={BUAN 448, HW3},
  hidelinks,
  pdfcreator={LaTeX via pandoc}}

\title{BUAN 448, HW3}
\author{}
\date{\vspace{-2.5em}2022-09-09}

\begin{document}
\maketitle

\hypertarget{question-1-shipments-of-household-appliances-line-graphs}{%
\subsection{Question 1: Shipments of Household Appliances: Line
Graphs}\label{question-1-shipments-of-household-appliances-line-graphs}}

The file ApplianceShipments.csv contains the series of quarterly
shipments (in millions of dollars) of US house- hold appliances between
1985 and 1989. Answer the following questions using the dataset:

\begin{itemize}
\tightlist
\item
  Item a) Create a well-formatted time plot of the data using ts and
  plot functions. (Hint: follow the same way we did in class) By looking
  at the plot, does there appear to be a quarterly pattern?
\end{itemize}

\begin{Shaded}
\begin{Highlighting}[]
\FunctionTok{library}\NormalTok{(ggplot2)}
\FunctionTok{library}\NormalTok{(forecast)}
\end{Highlighting}
\end{Shaded}

\begin{verbatim}
## Registered S3 method overwritten by 'quantmod':
##   method            from
##   as.zoo.data.frame zoo
\end{verbatim}

\begin{Shaded}
\begin{Highlighting}[]
\FunctionTok{library}\NormalTok{(here)}
\end{Highlighting}
\end{Shaded}

\begin{verbatim}
## here() starts at D:/A_Lehigh/2022 Fall/BUAN 488 - Predictive Analytics/HW/HW3
\end{verbatim}

\begin{Shaded}
\begin{Highlighting}[]
\NormalTok{shipment }\OtherTok{\textless{}{-}} \FunctionTok{read.csv}\NormalTok{(}\FunctionTok{here}\NormalTok{(}\StringTok{"ApplianceShipments.csv"}\NormalTok{))}


\NormalTok{shipment\_ts }\OtherTok{\textless{}{-}} \FunctionTok{ts}\NormalTok{(shipment}\SpecialCharTok{$}\NormalTok{Shipments, }\AttributeTok{start =} \FunctionTok{c}\NormalTok{(}\DecValTok{1985}\NormalTok{, }\DecValTok{1}\NormalTok{), }\AttributeTok{end =} \FunctionTok{c}\NormalTok{(}\DecValTok{1989}\NormalTok{, }\DecValTok{4}\NormalTok{), }\AttributeTok{frequency =} \DecValTok{4}\NormalTok{)}

\FunctionTok{plot}\NormalTok{(shipment\_ts, }\AttributeTok{xlab =} \StringTok{"Time"}\NormalTok{, }
     \AttributeTok{ylab =} \StringTok{"Shipment"}\NormalTok{, }
     \AttributeTok{ylim =} \FunctionTok{c}\NormalTok{(}\DecValTok{3500}\NormalTok{, }\DecValTok{5000}\NormalTok{), }
     \AttributeTok{main =} \StringTok{"Quarterly Shipments from 1985{-}1989"}\NormalTok{)}
\end{Highlighting}
\end{Shaded}

\includegraphics{HW3_files/figure-latex/code, Place your r code here for Q1 part a-1.pdf}
\#\#\#\# Question 1 Answer According to the previous plot, we can see
that the shipment from 1985 to 1989 shows a clear quarterly pattern.

\hypertarget{question-2-sales-of-riding-mowers-scatter-plots.}{%
\subsection{Question 2: Sales of Riding Mowers: Scatter
Plots.}\label{question-2-sales-of-riding-mowers-scatter-plots.}}

A company that manufactures riding mowers wants to identify the best
sales prospects for an intensive sales campaign. In particular, the
manufacturer is interested in classifying households as prospective
owners or nonowners on the basis of Income (in \$1000s) and Lot Size (in
1000 ft2). The marketing expert looked at a random sample of 24
households, given in the file RidingMowers.csv.

\begin{itemize}
\tightlist
\item
  Item a) Using R, create a scatter plot of Lot Size vs.~Income by plot
  function. By looking at the plot, do you get an insight?
\end{itemize}

\begin{Shaded}
\begin{Highlighting}[]
\NormalTok{mower }\OtherTok{\textless{}{-}} \FunctionTok{read.csv}\NormalTok{(}\FunctionTok{here}\NormalTok{(}\StringTok{"RidingMowers.csv"}\NormalTok{))}

\FunctionTok{plot}\NormalTok{(}\AttributeTok{y =}\NormalTok{ mower}\SpecialCharTok{$}\NormalTok{Income,}
     \AttributeTok{x =}\NormalTok{ mower}\SpecialCharTok{$}\NormalTok{Lot\_Size,}
     \AttributeTok{ylab =} \StringTok{"Mower Income (in $1000s)"}\NormalTok{,}
     \AttributeTok{xlab =} \StringTok{"Lot Size (in 1000 ft2)"}\NormalTok{,}
     \AttributeTok{main =} \StringTok{"Mowing Lot Size v.s. Mower Income"}\NormalTok{)}
\end{Highlighting}
\end{Shaded}

\includegraphics{HW3_files/figure-latex/code, Place your r code here for Q2 part a-1.pdf}
\#\#\#\# Question 2 Answer(a) According to the previous scatter plot, we
barely can see any clear relationship between lot size and income.

\begin{itemize}
\tightlist
\item
  Item b) Using R, create a scatter plot of Lot Size vs.~Income,
  color-coded by the outcome variable owner/nonowner by ggplot function
  (There is an argument for ``aes'' object that you can color points
  based on the variable you choose). This time you color the points by
  Ownership. Now, by looking at the plot, would you get a conclusion
  that there is a relationship between Lot\_Size and Ownership. Justify
  your answer.
\end{itemize}

\begin{Shaded}
\begin{Highlighting}[]
\FunctionTok{ggplot}\NormalTok{(mower) }\SpecialCharTok{+}
  \FunctionTok{geom\_point}\NormalTok{(}\FunctionTok{aes}\NormalTok{(}\AttributeTok{x =}\NormalTok{ Lot\_Size, }\AttributeTok{y =}\NormalTok{ Income, }\AttributeTok{color =}\NormalTok{ Ownership)) }\SpecialCharTok{+}
  \FunctionTok{facet\_wrap}\NormalTok{(}\SpecialCharTok{\textasciitilde{}}\NormalTok{Ownership, }\AttributeTok{ncol =} \DecValTok{2}\NormalTok{) }\SpecialCharTok{+}
  \FunctionTok{xlab}\NormalTok{(}\StringTok{"Lot Size (in 1000 ft2)"}\NormalTok{) }\SpecialCharTok{+}
  \FunctionTok{ylab}\NormalTok{(}\StringTok{"Mower Income (in $1000s)"}\NormalTok{) }\SpecialCharTok{+} 
  \FunctionTok{ggtitle}\NormalTok{(}\StringTok{"Relationship between Lot Size and Income}\SpecialCharTok{\textbackslash{}n}\StringTok{Comparsion between Different Ownership"}\NormalTok{)}
\end{Highlighting}
\end{Shaded}

\includegraphics{HW3_files/figure-latex/code, Place your r code here for Q2 part b-1.pdf}
\#\#\#\# Qestion 2 (b) Answer According to the previous graphs, we can
see that there's a relationship between lot size and ownership. When we
are looking at Owner, we can see the data points drop at high lot\_size
and high income, which concentrate at the upper right corner of the
graph. In contrast, the data points for non-owner concentrate at the
lower left part of the graph, which illustrates that non-owners usually
are not able to access to higher lot\_size and have high income.

\end{document}
